\section{Introduction}
\label{intro}

Temporal coding and rate coding are two major coding schemes adopted by neurons.
The paper~\cite{voe2007temporal} focuses on models of temporal coding. Specifically,
different from the previous studies where the shape of post-synaptic potentials (PSPs) is relevant for computations,
the author proposes a new theory based on the non-linear dynamics of integrate-and-fire neurons,
where the effect of synaptic currents depends on the internal state of the postsynaptic neuron, 
which is described by the Phase Response Curve (PRC). 
Based on the theory, the author uses theta neurons to construct a temporal coding model,
and shows in the paper that this model can be used to do some computation (PCA in the paper).
The advantage of the proposed model is that it does not rely on the shape of PSPs to do computation,
and the computations do not rely on the delays caused by inter-neuron transmissions.

In this report, we do not repeat model details and the derivation of the learning rule, as they have been described
detailedly in the paper. 
Instead, we focus on reviewing, explaining, and analyzing the model proposed in the paper.
We also highlight some key ideas behind the model design described in the paper.
We reimplement the programs and redo all the experiments mentioned in the paper,
besides some extra simulations that provide some detailed characterization of the model.
Most of the results are successfully reproduced by us (PRC of the theta model, model simulations, and PCA of a gaussian distribution), while we fail to reproduce the result of one experiment (encoding natural images).
We detailedly explain how we redo the experiments, analyze the results we obtain, and compare our results with
the author.
We also summarize some possible reasons of the mismatch of the experiment results and the limitations of the original work.

The remainder of this report is organized as follows:
We review, explain, and analyze the model including the theta neuron characterization and the derivation of the learning rule in Section~\ref{model}; 
We review the design of the auto-encoder network, describe how we redo the experiments conducted by the author,
and analyze the experiment results in Section~\ref{exp};
We discuss the problems we encounter and the limitations of the paper in Section~\ref{discussion};
Finally, we mark some errors we find in the paper in Section~\ref{error}.
